%%%%%%%%%%%%%%%%%%%%%%%%%%%%%%%%%%%%%%%%%%%%%%%%%%%%%%%%%%%%%%%%%%%%%%%%%%%%%%%%%
% a transition as an arrow interrupted by a symbol in a bracket
%%%%%%%%%%%%%%%%%%%%%%%%%%%%%%%%%%%%%%%%%%%%%%%%%%%%%%%%%%%%%%%%%%%%%%%%%%%%%%%%%%
\makeatletter
\DeclareRobustCommand{\shortto}{%
  \mathrel{\mathpalette\short@to\relax}%
}

\DeclareRobustCommand{\shortminus}{%
  \mathrel{\mathpalette\short@minus\relax}%
}

\newcommand{\short@to}[2]{%
  \mkern2mu
  \clipbox{{.5\width} 0 0 0}{$\m@th#1\vphantom{+}{\rightarrow}$}%
}

\newcommand{\short@minus}[2]{%
  \mkern2mu
  \clipbox{{.5\width} 0 0 0}{$\m@th#1\vphantom{+}{-}$}%
}
\makeatother

%%%%%%%%%%%%%%%%%%%%%%%%%%%%%%%%%
%variants of the transition arrow
%%%%%%%%%%%%%%%%%%%%%%%%%%%%%%%%%
%scriptstyle
%-------------------------------
%-[bla]->
%\newcommand{\labeledto}[1]{{{\shortminus}\hspace{-1.6pt}\raisebox{0.16ex}{$\scriptstyle[ #1\hspace{-0.28pt}]$}\hspace{-2.2pt}{\shortto}}}
%-<bla>->
%\newcommand{\labeledto}[1]{{{\shortminus}\hspace{-2pt}\raisebox{0.16ex}{$\scriptstyle\langle #1\hspace{-0.28pt}\rangle$}\hspace{-2.2pt}{\shortto}}}
%-{bla}->
%\newcommand{\labeledto}[1]{{{\shortminus}\hspace{-2pt}\raisebox{0.16ex}{$\scriptstyle\{ #1\hspace{-0.28pt}\}$}\hspace{-2.2pt}{\shortto}}}
%-(bla)->
%\newcommand{\labeledto}[1]{{{\shortminus}\hspace{-1.0pt}\raisebox{0.16ex}{$\scriptstyle( #1\hspace{-0.28pt})$}\hspace{-1.6pt}{\shortto}}}
%-------------------------------
%small
%-------------------------------
%-[bla]->
%\newcommand{\labeledto}[1]{{{\shortminus}\hspace{-1.6pt}\raisebox{0.16ex}{$\scriptstyle[ #1\hspace{-0.28pt}]$}\hspace{-2.2pt}{\shortto}}}
%-<bla>->
%\newcommand{\labeledto}[1]{{{\shortminus}\hspace{-2pt}\raisebox{0.16ex}{$\scriptstyle\langle #1\hspace{-0.28pt}\rangle$}\hspace{-2.2pt}{\shortto}}}
%-{bla}->
%\newcommand{\labeledto}[1]{{{\shortminus}\hspace{-2pt}\raisebox{0.16ex}{$\scriptstyle\{ #1\hspace{-0.28pt}\}$}\hspace{-2.2pt}{\shortto}}}
%-(bla)->
\newcommand{\labeledto}[1]{\raisebox{-0.2pt}{\scalebox{1.2}{\ensuremath{{\shortminus}\hspace{-1.0pt}\raisebox{0.16ex}{$\scriptstyle(#1\hspace{-0.28pt})$}\hspace{-1.75pt}{\shortto}}}}}


%%%%%%%%%%%%%%%%%%%%%%%%%%%%%%%%%
%automata arrow in superscript and subscritp script, so far only the -{bla}->
%%%%%%%%%%%%%%%%%%%%%%%%%%%%%%%%%
\newcommand{\scriptlabeledto}[1]{{{\shortminus}\hspace{-1.0pt}\raisebox{0.12ex}{$\scriptscriptstyle\{ #1\hspace{-0.28pt}\}$}\hspace{-1.6pt}{\shortto}}}

\def\moveStrange(#1,#2,#3){%somewhat shorter syntax 
\mathchoice
{#1\,\labeledto{#2}\,#3}
{#1\labeledto{#2}#3}
{#1\scriptlabeledto{#2}#3}
{#1\scriptlabeledto{#2}#3}
}
\newcommand\move[3]{%standard LaTex syntax where each parameter is put in {}
\mathchoice
{#1\,\labeledto{#2}\,#3}
{#1\labeledto{#2}#3}
{#1\scriptlabeledto{#2}#3}
{#1\scriptlabeledto{#2}#3}
}

%%%%%%%%%%%%%%%%%%%%%%%%%%%%%%%%%%%%%%%%%%%%%%%%%%%%%%%%%%%%%%

%commenting
\newcommand{\td}[1]{\textcolor{blue}{\ifmmode \text{[#1]}\else [#1] \fi}}
\newcommand{\ol}[1]{\textcolor{blue}{\ifmmode \text{[OL: #1]}\else [OL: #1] \fi}}
\newcommand{\tv}[1]{\textcolor{magenta}{\ifmmode \text{[TV: #1]}\else [TV: #1] \fi}}
\newcommand{\lh}[1]{\textcolor{orange}{\ifmmode \text{[LH: #1]}\else [LH: #1] \fi}}
\newcommand{\lt}[1]{\textcolor{green}{\ifmmode \text{[LT: #1]}\else [LT: #1] \fi}}
%\newcommand{\lh}[1]{\todo[color=green!20]{LH: #1}}
\newcommand{\mv}[1]{\textcolor{purple}{\ifmmode \text{[MV: #1]}\else [MV: #1] \fi}}
\newcommand{\os}[1]{\textcolor{cyan}{\ifmmode \text{[OS: #1]}\else [OS: #1] \fi}}
%\newcommand{\remove}[1]{\textcolor{red}{\ifmmode \text{#1}\else #1 \fi}}
%\newcommand{\remove}[1]{\textcolor{red}{\ifmmode \text{RM}\else RM \fi}}
\newcommand{\focus}[1]{\textcolor{red}{\ifmmode \text{#1}\else #1 \fi}}

%\renewcommand{\td}[1]{}
%\renewcommand{\ol}[1]{}
%\renewcommand{\tv}[1]{}
%\renewcommand{\lh}[1]{}
%\renewcommand{\mv}[1]{}
%\renewcommand{\lt}[1]{}

\newcommand{\regex}[1]{\ensuremath{\texttt{#1}}}

% Blinding
\newcommand{\blinded}[1]{\ifx\blindreview\undefined #1 \else \textcolor{black!65}{[blinded for review]}\fi}

%general
\newcommand{\nat}{\mathbb{N}}

\newcommand{\secref}[1]{\S{\ref{#1}}}

%Transition systems
\newcommand{\system}{T} %trnasition system
\newcommand{\var}{v} %configuration variable
\newcommand{\vars}{V} %all configuration variables
% \newcommand{\svar}{x} %symbol variable
% \newcommand{\svar}{\alpha} %symbol variable
% \newcommand{\svar}{\chi} %symbol variable
%\newcommand{\svar}{\mathit{in}} %symbol variable
% \newcommand{\svar}{\boldsymbol{\ell}} %symbol variable
% \newcommand{\svar}{\ell} %symbol variable
% \newcommand{\svar}{\mathbbmss{l}} %symbol variable
% \newcommand{\svar}{\mathbbmtt{l}} %symbol variable
% \newcommand{\svar}{\mathbbm{l}} %symbol variable
\newcommand{\svar}{\texttt{l}} %symbol variable
\newcommand{\domain}{\mathbb{D}} %variable domain
\newcommand{\symbols}{\Sigma} %lphabet
\newcommand{\symb}{a} %alphabet symbol
\newcommand{\conf}{\alpha} %configuration of on LTS 
\newcommand{\confs}{\mathbb{C}} %all configurations
% \newcommand{\assignments}{f} %assignments on a transition
\newcommand{\assignments}{\mathit{f}} %assignments on a transition
\newcommand{\psphere}{\psi} %parametric sphere
\newcommand{\normpsphere}{\psphere_n} %normalized sphere
\newcommand{\semof}[1]{\llbracket#1\rrbracket} %semantics of something
\newcommand{\val}{\nu} %general valuation of variables
% \newcommand{\preds}[0]{\mathbb{P}}   % symbol predicates
\newcommand{\preds}[0]{\mathcal{P}}   % symbol predicates
\newcommand{\theory}[0]{\mathbb{T}}  % theory of counters

%counting automata 
\newcommand{\automaton}{A} %counting automaton

\newcommand{\autConf}{A_{conf}} 
\newcommand{\aut}[0]{\automaton}
\newcommand{\autof}[1]{\aut_{#1}}
\newcommand{\states}{Q} 
\renewcommand{\state}{q}
\newcommand{\stateone}{q}

\newcommand{\statetwo}{r}
\newcommand{\run}{\phi}
\newcommand{\labels}{labels}
%\newcommand{\stvar}{s} %state variable
%\newcommand{\stvar}{\boldsymbol{s}} %state variable
%\newcommand{\stvar}{\mathbbmtt{s}} %state variable
%\newcommand{\stvar}{\mathbbmss{s}} %state variable
%\newcommand{\stvar}{\mathbbm{s}} %state variable
\newcommand{\stvar}{\texttt{s}} %state variable
% \newcommand{\stvar}{\mathfrak{s}} %state variable
\newcommand{\stvars}{S} %all state variables
\newcommand{\trel}{\Delta} %transition formula
\newcommand{\cvar}{c} %counters 
\newcommand{\cvars}{C} %all counters
\newcommand{\sguard}{\alpha} %symbol guard on a transition
\newcommand{\cguard}{\beta} %counter guard on a trantision
%\newcommand{\guard}{\gamma}
\newcommand{\guardnew}{h} %guard on atransition (mixed symbols and counuters) [new one]
%\newcommand{\update}{\mathit{update}} %combinatino of assignments in the factorised form
\newcommand{\tf}{\phi} %transition formula
% States in transitions as blue bubbles
%\tikzset{mstatestyle/.style={draw,rectangle,rounded corners,fill=blue!30,inner xsep=.3em,inner ysep=0em,text height=2ex,text depth=.9ex}}
% States in transitions as white bubbles
\tikzset{mstatestyle/.style={draw,rectangle,rounded corners,fill=blue!0,inner xsep=.3em,inner ysep=0em,text height=2ex,text depth=.9ex}}
\newcommand{\mstate}[1]{\tikz[baseline,anchor=base]{\node[mstatestyle] {\ensuremath{#1}};}} %macrostate
\newcommand{\limpl}[0]{\mathop{\rightarrow}}

%formulae, substitutions, mappings
\newcommand{\subst}{\theta} %substitution
\newcommand{\substs}{\Theta} %set of substitutions
\newcommand{\substc}{\subst_{\mathit{const}}} %subst of integer constants by pars
\newcommand{\substa}{\subst_{\mathit{at}}} %... of assignment terms ...
\newcommand{\substp}{\subst_{\mathit{par}}} %... of parameters ...
\newcommand{\unprime}{\subst_{\mathit{unprime}}} %subst that removes prime
\newcommand{\rename}{\subst_{\mathit{rename}}} %subst that renames parameters
%\newcommand{\renamePrimed}{\subst_{\mathit{rename}}^{\mathit{primed}}} %subst that renames primed parameters
\newcommand{\renamePrimed}{\subst'_{\mathit{rename}}} %subst that renames primed parameters
\newcommand{\renames}{\substs_{\mathit{rename}}} %set of renamings
\newcommand{\domof}[1]{\mathbf{dom}(#1)} %domain of a mapping
\newcommand{\imgof}[1]{\mathbf{img}(#1)} %image of a mapping
\newcommand{\Pars}{\mathbb{P}} %pool of parameters (unbounded perhaps)
\newcommand{\Parsof}[1]{\Pars_{|#1}} %pool of parameters (unbounded perhaps) of
\newcommand{\restr}[2]{#1|_{#2}} %restric the domain of a map
\newcommand{\subs}[1]{[#1]} %substitution
%\newcommand{\elim}[0]{\mathit{elim}} % quantifier elimination
\newcommand{\elim}[0]{\mathit{elim}} % quantifier elimination
%\newcommand{\elimof}[2]{\elim(\exists #1:#2)} % quantifier elimination of
%\newcommand{\elimof}[2]{{\boldsymbol{\exists}} #1:#2} % quantifier elimination of
%\newcommand{\elimof}[2]{{\exists\hspace{-3.7pt}\exists} #1:#2} % quantifier elimination of
% \newcommand{\elimof}[2]{{\exists\hspace{-3.7pt}\exists\hspace{-3.8pt}\exists} #1:#2} % quantifier elimination of
% \newcommand{\elimof}[2]{{\exists\hspace{-1.5mm}\exists\hspace{-1.5mm}\exists} #1:#2} % quantifier elimination of
\newcommand{\elimof}[2]{{\mathrlap{\mathrlap{\hspace*{0.7mm}\exists}{\hspace*{0.35mm}\exists}}{\exists\hspace{0.8mm}}} #1:#2} % quantifier elimination of
%\newcommand{\issat}[0]{\texttt{SAT}}
\newcommand{\issat}[0]{\texttt{sat}}
\newcommand{\issatof}[1]{\issat(#1)}
\newcommand{\difcnts}[0]{{\neq}}
\newcommand{\difcntsof}[1]{\;\difcnts(#1)}
\newcommand{\inrange}[0]{\mathit{range}}
\newcommand{\inrangeof}[1]{\inrange(#1)}
% \newcommand{\defeq}[0]{\mathop{\stackrel{\textrm{\tiny def}}{=}}}
\newcommand{\defeq}[0]{\mathrel{\stackrel{\textrm{\tiny def}}{=}}}
% \newcommand{\defequiv}[0]{\mathop{\stackrel{\mathrm{def}}{\equiv}}}
%\newcommand{\defequiv}[0]{\triangleq}
\newcommand{\defequiv}{\defeq}

%pseudocode
\newcommand{\wl}{\mathit{Worklist}} 
\newcommand{\wlt}{\mathit{Candidates}} %worklist of transitions
\newcommand{\ass}{\leftarrow} %assignment
\newcommand{\processed}{\psphere} %spehre taken from the worklist
\newcommand{\src}{\psphere_\mathit{src}} %source pspehere of a trans
\newcommand{\tgt}{\psphere_\mathit{tgt}} %target ...
\newcommand{\old}{\psphere_\mathit{old}} %old psphere equivalent to a new one


%regex and derivative related
\newcommand{\REpreds}[1]{\textit{Preds}(#1)}
\newcommand{\REV}[1]{#1^{\mathbf{r}}}
\newcommand{\pair}[2]{\langle #1, #2 \rangle}
\newcommand{\triple}[3]{\langle #1, #2, #3\rangle}
\newcommand{\pred}[1]{#1}
\newcommand{\eps}{\epsilon}
\newcommand{\concat}{\cdot}
\newcommand{\den}[2][{}]{[\![#2]\!]_{#1}}
\newcommand{\Max}{\textit{\textbf{max}}}
\newcommand{\Min}{\textit{\textbf{min}}}
\newcommand{\CD}[3][{}]{\partial_{#2}^{#1}(#3)}
\newcommand{\Minterms}[1]{\textit{Minterms}(#1)}
\newcommand{\CounterGuard}[1]{\guard(#1)}
\newcommand{\CounterUpdate}[2][{}]{\boldsymbol{#2}_{\!\!#1}}
\newcommand{\CounterUpdateFormula}[1]{\varphi_{#1}}
\newcommand{\NullCond}[1]{\boldsymbol{\epsilon}_{#1}}
\newcommand{\subcntr}{\prec}
%\renewcommand{\max}{\Max}
\newcommand{\seqcomp}{\mathbb{;\,}}

\newcommand{\True}{\mathit{true}}
\newcommand{\False}{\mathit{false}}
\newcommand{\ResetOp}[1][{}]{\textsc{set0}_{#1}}
\newcommand{\SetOneOp}{\textsc{set1}}
\newcommand{\InitZeroOp}{\textsc{init0}}
\newcommand{\InitOneOp}{\textsc{init1}}
%\newcommand{\InitOneOp}{\textsc{init}{\circ}\textsc{incr}}
\newcommand{\ResetZeroOp}{\textsc{res0}}
\newcommand{\ResetOneOp}{\textsc{res1}}
\newcommand{\ExitInitOp}{\ExitOp{\circ}\InitOp}
%\newcommand{\ExitIncrOp}{\ExitOp{\circ}\IncrOp}
%\newcommand{\ExitIncrOp}{\ExitOp\mathtt{;}\IncrOp}
\newcommand{\InitIncrOp}{\InitOp{\circ}\IncrOp}
\newcommand{\AddZeroOp}{\textsc{add0}}
%\newcommand{\AddOneOp}{\AddZeroOp{\circ}\IncrOp}
\newcommand{\AddOneOp}{\textsc{add1}}

%\newcommand{\IncrCSOp}[1][{}]{\textsc{incr}_{#1}}
%\newcommand{\ExitCSOp}[1][{}]{\textsc{rst}_{#1}}
%\newcommand{\ExitIncrCSOp}[1][{}]{\textsc{rst1}_{#1}}
%\newcommand{\NoCSOp}[1][{}]{\textsc{noop}_{#1}}
%\newcommand{\IncrCSOp}[1][{}]{\textsc{csIncr}_{#1}}
%\newcommand{\ExitCSOp}[1][{}]{\textsc{csExit}_{#1}}
%\newcommand{\ExitIncrCSOp}[1][{}]{\textsc{csExit1}_{#1}}
%\newcommand{\NoCSOp}[1][{}]{\textsc{csNoop}_{#1}}
%\newcommand{\IncrCSOp}[1][{}]{\textsc{Incr}_{#1}}
%\newcommand{\ExitCSOp}[1][{}]{\textsc{Reset0}_{#1}}
%\newcommand{\ExitIncrCSOp}[1][{}]{\textsc{Reset1}_{#1}}
%\newcommand{\NoCSOp}[1][{}]{\textsc{Noop}_{#1}}



\renewcommand{\Reset}[1]{#1{\mapsto}\ExitOp}
\newcommand{\SetOne}[1]{#1{\mapsto}\SetOneOp}
\newcommand{\Inv}[1]{\varphi_{#1}}

\newcommand{\spair}[2]{\{\pair{#1}{#2}\}}
\newcommand{\striple}[3]{\{\triple{#1}{#2}{#3}\}}

%\newcommand{\id}{\textit{\O}}
\newcommand{\id}{\textbf{\textsc{id}}}
\newcommand{\ONE}{\spair{\id}{\emp}}

\newcommand{\A}{\texttt{a}}
\newcommand{\nA}{\texttt{[{\caret}a]}}
\newcommand{\Loop}[2]{\texttt{#1\{\begin{math}#2\end{math}\}}}

%\newcommand{\RESET}[1]{\textit{skip}(#1)}
\newcommand{\NULL}[1]{\textit{null}(#1)}
\newcommand{\emp}{\varepsilon}

%special symbols
\newcommand{\bslash}{\char`\\}
\newcommand{\caret}{\char`\^}
\newcommand{\underscore}{\char`\_}
\newcommand{\leftbrace}{\char`\{}
\newcommand{\rightbrace}{\char`\}}
\newcommand{\tildechar}{\char`~}


%equality by definition
\newcommand{\eqdef}{\stackrel{\raisebox{-3pt}{{\textrm{\tiny def}}}}{=}}
%the above one increases spaces between lines, ugly
%\newcommand{\eqdef}{\mathchoice{\stackrel{\textrm{\tiny def}}{=}}{=_{\textrm{\tiny def}}}{\stackrel{\textrm{\tiny def}}{=}}{\stackrel{\textrm{\tiny def}}{=}}}


%Boolean algebra
\newcommand{\DOT}{\texttt{\makebox[4pt]{{\huge\texttt{{.}}}}}}
\newcommand{\BA}{\mathcal{A}}
\newcommand{\FALSE}{\texttt{[]}}

%standard types
\newcommand{\Bool}{\mathbb{B}}
\newcommand{\Nat}{\mathbb{N}}


\newcommand\termsof[1]{\mathit{terms}(#1)}
\newcommand\substf[1]{\varphi_{#1}}
\newcommand\init{I}
\newcommand\final{F}
\newcommand\outcome{\mathit{out}}
\newcommand\successor[3]{#1\xrightarrow{#2}#3}
\newcommand\num[1]{\mathit{num}(#1)}
\newcommand\at[1]{\mathit{at}(#1)}

\newcommand{\cnt}[0]{\mathit{cnt}}
\newcommand{\detaut}[0]{\automaton^{d}}
\newcommand{\detautof}[1]{\detaut_{#1}}
\newcommand{\autcc}[0]{\automaton_{\mathit{cc}}}    %aut. for char. classes
\newcommand{\detautcc}[0]{\detaut_{\mathit{cc}}}    %det. aut. for char. classes
% \newcommand{\stateofcnt}[0]{\mathit{state}}
\newcommand{\stateofcnt}[0]{\sigma}
\newcommand{\stateofcntof}[1]{\stateofcnt(#1)}
\newcommand{\setofspof}[1]{\lbag #1 \rbag}

\newcommand{\statesdom}[0]{\mathbb{Q}}

\newcommand{\re}{{r.e.}}
\newcommand{\DOM}{\mathfrak{D}}
\newcommand{\PRED}{\Psi}
\newcommand{\SAT}{\textit{\textbf{Sat}}}
\newcommand{\BDDA}{\mathcal{B}}
\newcommand{\BIT}[1]{\beta_{#1}}

\newcommand{\MIN}[1]{\textit{\textbf{min}}_{#1}}
\newcommand{\MAX}[1]{\textit{\textbf{max}}_{#1}}


%%%%%%%%%%%%%%%%%%%%%%%%%%%%%%%%%%%%%%%%%%%%%%%%%%%
% USED SET AND COUNTER OPERATIONS AND ETSTS
%%%%%%%%%%%%%%%%%%%%%%%%%%%%%%%%%%%%%%%%%%%%%%%%%%%
%\newcommand{\CanExit}[1]{\xi_{#1}}
%\newcommand{\CanIncr}[1]{\iota_{#1}}
\newcommand{\CanIncr}[1]{\textsc{canIncr}_{#1}}

\newcommand{\Op}[1][{}]{\IndexedOp{\textsc{op}}{#1}}
\newcommand{\IncrOp}[1][{}]{\IndexedOp{\textsc{incr}}{#1}}
\newcommand{\NoOp}[1][{}]{\IndexedOp{\textsc{noop}}{#1}}

%Version one
%\newcommand{\CanExit}[1]{\textsc{canRst}_{#1}}
%\newcommand{\ExitOp}[1][{}]{\IndexedOp{\textsc{rst}}{#1}}
%\newcommand{\ExitIncrOp}[1][{}]{\IndexedOp{\textsc{rst1}}{#1}}
%\newcommand{\IncrCSOp}[1][{}]{\textsc{incr}_{#1}}
%\newcommand{\ExitCSOp}[1][{}]{\textsc{rst}_{#1}}
%\newcommand{\ExitIncrCSOp}[1][{}]{\textsc{rst1}_{#1}}
%\newcommand{\NoCSOp}[1][{}]{\textsc{noop}_{#1}}
%\newcommand{\CanExitS}[1]{\textsc{canRst}_{#1}}
%\newcommand{\CanIncrS}[1]{\iota_{#1}}

%Varsion two
\newcommand{\CanExit}[1]{\textsc{canExit}_{#1}}
\newcommand{\ExitOp}[1][{}]{\IndexedOp{\textsc{exit}}{#1}}
\newcommand{\ExitIncrOp}[1][{}]{\IndexedOp{\textsc{exit1}}{#1}}
\newcommand{\IncrCSOp}[1][{}]{\textsc{Incr}_{#1}}
\newcommand{\ExitCSOp}[1][{}]{\textsc{Rst}_{#1}}
\newcommand{\ExitIncrCSOp}[1][{}]{\textsc{Rst1}_{#1}}
\newcommand{\NoCSOp}[1][{}]{\textsc{Noop}_{#1}}
\newcommand{\CanExitS}[1]{\textsc{canExit}_{#1}}
\newcommand{\CanIncrS}[1]{\textsc{canIncr}_{#1}}
%
% The idea here is that set operations, mainly Reset, are not directly connected
% to the counter operations of the same names. Set is reset when counters exit 
% or when their scope is entered. 
% The conflict "cannot exit but exitting" does not seem visible here
% \xi still looks similar to canExit

%%%%%%%%%%%%%%%%%%%%%%%%%%%%%%%%%%%%%%%%%%%%%%%%%%%
%%%%%%%%%%%%%%%%%%%%%%%%%%%%%%%%%%%%%%%%%%%%%%%%%%%

\newcommand{\SetZeroOP}[1]{\textit{Set0}_{#1}}
\newcommand{\SetOneOP}[1]{\textit{Set1}_{#1}}
\newcommand{\IncrOP}[1]{\textit{Incr}_{#1}}
\newcommand{\NOOP}[1]{\textit{NOOP}_{#1}}


\newcommand{\BDDAxA}{\BDDA\langle{\AA}\rangle}

\newcommand{\PlusOne}[1]{\textit{\textbf{incr}}_{#1}}
\newcommand{\MaxOf}[1]{\textit{Max}(#1)}
\newcommand{\MinOf}[1]{\textit{Min}(#1)}
\newcommand{\PlusOneZ}[1]{\textit{\textbf{incr}}_{#1}^0}

\newcommand{\CounterOf}{\textit{Ctr}}

\newcommand{\ExitTag}{\textsc{rst}}
\newcommand{\IncrTag}{\textsc{incr}}
\newcommand{\NoopTag}{\textsc{noop}}
\newcommand{\InitTag}{\textsc{init}}
\newcommand{\IncrInitTag}{\textsc{incr0}}
\newcommand{\Tag}{\textsc{tag}}
\newcommand{\DFA}{\textsc{dfa}}

\newcommand{\NFAof}[1]{\textit{FA}(#1)}
\newcommand{\DFAof}[1]{\textit{DFA}(#1)}
\newcommand{\IFAof}[1]{\textit{s-IA}(#1)}

\newcommand{\RANGE}[1]{\textit{InRange}_{#1}}

\newcommand{\CountersOf}{\textit{\textbf{cs}}}

\newcommand{\FinCond}[1]{F\ifthenelse{\equal{#1}{}}{}{(#1)}}


\newcommand{\IfThenElse}[3]{\textit{\textbf{if}}\;#1\;\textit{\textbf{then}}\;#2\;\textit{\textbf{else}}\;#3}

\newcommand{\Init}{\textit{Init}}

\newcommand{\OPS}{\mathcal{O}}

\newcommand{\CVType}[1]{\mathfrak{#1}}
\newcommand{\CUType}[1]{\OPS_{#1}}
\newcommand{\CounterPreds}[1]{\Psi_{#1}}

\newcommand{\CS}[1]{#1'}
%\newcommand{\CS}[1]{\ddot {#1}}
%\newcommand{\CS}[1]{\widehat{#1}}
%\newcommand{\CS}[1]{\hat{#1}}
%\newcommand{\CS}[1]{\bar {#1}}
%\newcommand{\CS}[1]{\tilde{#1}}
%\newcommand{\CS}[1]{#1^{\mathsf{s}}}
%\newcommand{\CS}[1]{#1^{\circ}}
%\newcommand{\CS}[1]{\mathfrak{#1}}
%\newcommand{\CS}[1]{\mathbbm{#1}}
%\newcommand{\CS}[1]{\mathbbmss{#1}}
%\newcommand{\CS}[1]{\bar {#1}}
%\newcommand{\CS}[1]{\bar {#1}}
%\newcommand{\CS}[1]{\dot {#1}}
%\newcommand{\CS}[1]{{#1}^{\circledS}}
%\newcommand{\CS}[1]{{#1}^{\{\}}}

\renewcommand{\AA}{\mathbb{A}}
\newcommand{\II}{\mathbb{I}}
\newcommand{\CC}{\mathbb{C}}
%\newcommand{\CCS}{\mathbb{C}_{\!\!\!s}}
%\newcommand{\CCS}{\mathfrak{C}}
\newcommand{\CCS}{\mathbb{S}}
%\newcommand{\CCS}{\mathbb{\CS{C}}}
\newcommand{\BB}{\mathbb{B}}
%\newcommand{\AACCS}{\mathbb{A}{\cdot}\CCS}
\newcommand{\AACCS}{\mathbb{A}{\times}\CCS}
%\newcommand{\IICCS}{\mathbb{I}{\cdot}\CCS}
\newcommand{\IICCS}{\mathbb{I}{\times}\CCS}
\newcommand{\AACC}{\AA{\cdot}\CC}
\newcommand{\IICC}{\II{\cdot}\CC}
\newcommand{\AB}{\mathbb{A}{\times}\mathbb{B}}

\newcommand{\Den}[1]{\boldsymbol{#1}}
\newcommand{\CSA}{CsA\xspace}
\newcommand{\CA}{\textrm{CA}\xspace}

%\newcommand{\prodp}[2]{#1{\cdot}#2}
\newcommand{\prodp}[2]{#1{\wedge}#2}

\newcommand{\mem}{\mathfrak{m}}

\newcommand{\smem}{\mathfrak{s}}

\newcommand{\PowerSet}[1]{\mathcal{P}(#1)}
%\newcommand{\PowerSetNE}[1]{\mathcal{P}_0(#1)}
\newcommand{\PowerSetNE}[1]{\mathcal{P}_{\text{fin}}(#1)}

\newcommand{\PowerState}{S}
\newcommand{\CountingSet}{V}

\newcommand{\PreCond}[1]{\varphi_{#1}}

\newcommand{\scope}{\sigma}
%\newcommand{\TrOps}{\omega}
\newcommand{\TrOps}{\textsc{op}}
\newcommand{\undefVal}{\bot}
\newcommand{\bigo}{\mathcal O}
\newcommand{\bigOof}[1]{\bigo(#1)}
%\newcommand{\CO}{\textsc{Op}}
%\newcommand{\CO}{\CS{op}}
\newcommand{\CO}{\textsc{op}}


\newcommand{\precond}{\phi^{\textrm{pre}}}


\newcommand{\decl}{\sigma^{\textrm{\tiny{-1}}}}

\newcommand{\Iff}{\Longleftrightarrow}


%\newcommand{\fincond}[1]{\varphi^{\textrm{fin}}_{#1}}
\newcommand{\fincond}[1]{\textit{fin}(#1)}

\newcommand{\chara}{\texttt{a}}


\newenvironment{ex}{\begin{example}}{
\qed
\end{example}}

\newenvironment{exnoqed}{\begin{example}}{
\end{example}}

\newcommand{\consistent}{\Delta_{R,\alpha , \beta}}
\newcommand{\incoming}{\mathit{incoming}}
\newcommand{\interesting}{\mathit{interesting}}
\newcommand{\nonzero}{\scope}

\newcommand{\vc}[1]{\begin{tabular}{@{}l@{}}#1\end{tabular}}


\newcommand{\CharClass}{\textit{CharClass}}


\newcommand{\bs}[1]{\mathbf{#1}}

%\newcommand{\grep}[0]{\texttt{grep}\xspace}
%\newcommand{\retwo}[0]{\textsc{RE2}\xspace}

\newcommand{\grep}[0]{\texttt{grep}\xspace}
\newcommand{\retwo}[0]{\texttt{RE2}\xspace}
\newcommand{\srm}[0]{\texttt{SRM}\xspace}
\newcommand{\dotnet}[0]{\texttt{.NET}\xspace}
\newcommand{\catool}[0]{\texttt{Cavil}\xspace}
\newcommand{\FA}{FA\xspace} %I changed s-FA to FA, since we do not use FA, and CA, CsA is wihout s-, and s-FA is a bit heavy. But it can be changed back.
\newcommand{\timeout}[0]{TO\xspace}

% \newcommand{\derivpic}[1]{(#1)}
\newcommand{\derivpic}[1]{\lbag #1 \rbag}


\newcommand{\incr}[2]{#1(#2{+}{+})}


\newcommand{\DERIV}[3]{\boldsymbol{\partial}^{#1}_{#2}(#3)} %derivative



%---- (possibly indexed) operation macros

\newcommand{\IndexedOp}[2]{#1_{#2}}


\newcommand{\InitOp}[1][{}]{\textsc{init}_{#1}}


\newcommand{\update}[1]{\mathit{upd}(#1)}
\newcommand{\Update}[1]{\mathbf{#1}}
\newcommand{\guard}[1]{\mathit{grd}(#1)}
\newcommand{\Guard}[1]{\varphi_{#1}}


\newcommand{\BraceIfThenElse}[3]{
\left\{
\begin{array}{@{}ll@{}}
#2 & \textrm{if}\; #1 \\
#3 & \textrm{otherwise}
\end{array}
\right.
}


\newcommand{\blambda}{\boldsymbol{\lambda}}

\newcommand{\Qnca}[0]{Q_{N}}

\newcommand{\rewrites}{\leadsto}


\newcommand{\LPAR}{\textrm{\Large(}}
\newcommand{\RPAR}{\textrm{\Large)}}


\newcommand{\VISIBLE}[1]{\textit{Visible}(#1)}
\newcommand{\HIDDEN}[1]{\textit{Hidden}(#1)}
\newcommand{\COUNTERS}[1]{\textit{Counters}(#1)}
\newcommand{\RESET}[1]{\textbf{rst}_{#1}}

%language in state q
%\newcommand{\Lq}[2][{}]{\mathscr{L}_{#1}(#2)}
\newcommand{\Lq}[2][{}]{\mathcal{L}_{#1}(#2)}

\newcommand{\lang}[0]{\mathcal{L}} %language
\newcommand{\langof}[1]{\lang(#1)} %language of

\newcommand{\memInit}{\boldsymbol{0}}

\newcommand{\LANG}[2]{\mathbf{L}^{#1}(#2)} %language relative to counter state


\newcommand{\CDp}[2][+]{\partial^{#1}(#2)}

% \newcommand{\fragmenteasy}[0]{\textsc{Fragment}\xspace}
\newcommand{\fragmenteasy}[0]{\textsc{Aligned}\xspace}
\newcommand{\fragmenteqlen}[0]{$k$-\textsc{adic}\xspace}
\newcommand{\condof}[1]{\textsc{Cond}$_{#1}$}
% \newcommand{\unmark}[0]{\mathit{unmark}}
% \newcommand{\unmarkof}[1]{\unmark(#1)}
\newcommand{\unmark}[0]{\unmarkof \cdot}
\newcommand{\unmarkof}[1]{\lfloor #1 \rfloor}


\newcommand{\SIZEOF}[1]{\sharp(#1)}
\newcommand{\LEAFCNT}[1]{\sharp_{\Psi}(#1)}

\newcommand{\figtrans}[2]{#1 / #2}
\newcommand{\figtranstwoline}[2]{\begin{array}{c}#1 /\\ #2\end{array}}

\newcommand{\pstateof}[1]{#1^\mathit{DFA}}
\newcommand{\notacl}{\regex{[\string^a]}}
\newcommand{\acl}{\regex{a}}

% continuation of regex in the table
\newcommand{\recont}[0]{\textcolor{gray}{$\sim$}}
\newcommand{\mrtwo}[1]{\multirow{2}{*}{#1}}

