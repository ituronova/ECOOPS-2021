%%%%%%%%%%%%%%%%%%%%%%%%%%%%%%%%%%%%%%%%%%%%%%%%%%%%%%%%%%%%%%%%%%%%%%%%%%%%%%%%
\section{Implementation}\label{sec:Implementation}
%%%%%%%%%%%%%%%%%%%%%%%%%%%%%%%%%%%%%%%%%%%%%%%%%%%%%%%%%%%%%%%%%%%%%%%%%%%%%%%%

We discuss some important implementation details relating to the
implementations of $\II$, $\CC$, $\CCS$, and the product algebra
$\IICCS$, and their role in the {\CA} determinization algorithm.
All algorithms are implemented in C\# using the open source Microsoft.Automata
library~\cite{automata_library}.

%------------------------------------------------------------------------------
\paragraph{BDD algebra.}
%------------------------------------------------------------------------------

We use an effective Boolean algebra $\BDDA$ that we call a
\emph{BDD algebra} with universe $\nat$. The basic
predicates of $\BDDA$ have the form $\BIT{i}$ where $i\geq 0$ is a bit
position and $\den[\BDDA]{\BIT{i}}$ is the set of all
$n$ whose $i$'th bit in binary is 1.  $\PRED_\BDDA$ consists of BDDs
and its Boolean operations are corresponding BDD operations.  We write
$|$ for $\vee_{\BDDA}$, $\&$ for $\wedge_{\BDDA}$, and
$\overline{\psi}$ for $\neg_{\BDDA}\psi$.

%------------------------------------------------------------------------------
\paragraph{Algebras $\CC$ and $\CCS$.}
%------------------------------------------------------------------------------

We use $\BDDA$ to encode predicates over a given collection of counters
$c_i$ for $0\leq i \leq k$ as follows.
We represent the condition $\CanExit{i}$ with the $\BDDA$-predicate
$\BIT{2i}$ and the condition $\CanIncr{i}$ with the $\BDDA$-predicate
$\BIT{2i+1}$. Thus, assuming $c_i$ is in its valid range, in $\CC$,
\begin{itemize}
  \item $\BIT{2i}\& \BIT{2i+1}$ represents the case $\MIN{i} \leq c_i <  \MAX{i}$,
  \item $\BIT{2i}\& \overline{\BIT{2i+1}}$ represents the case $c_i =  \MAX{i}$,
  \item $\overline{\BIT{2i}}\& \BIT{2i+1}$ represents the case $c_i <  \MIN{i}$.
\end{itemize}
The predicate $\overline{\BIT{2i}}\& \overline{\BIT{2i+1}}$ represents
the impossible condition that $c_i < \MIN{i}$ and $\MAX{i} \leq c_i$.
Therefore, the predicate $\BIT{2i}|\BIT{2i+1}$ is used to elimiate the
impossible case and is always asserted in conjunction with any counter
predicate for $c_i$.
Analogously for $\CCS$ because we work with nonempty sets.

%------------------------------------------------------------------------------
\paragraph{Input algebra $\II$.}
%------------------------------------------------------------------------------

Given a collection of character classes, obtained from a regex $R$, we
first compute all the minterms over those character classes. Let the
minterms be $\Sigma = \{\alpha_i\}_{i<k}$. It turns out that not only is there
no explosion in the size of $\Sigma$ but in almost all
cases $k\leq 64$.  Due to the low value of $k$, each
minterm $\alpha_i$ in $\PRED_{\II}$ is represented internally by the
number $2^i$ and conjunction of predicates is bit-wise-and, complement
is bit-wise-complement of numbers, where the type of those numbers
is typically \texttt{UInt64}.


%------------------------------------------------------------------------------
\paragraph{Product algebra $\IICCS$.}
%------------------------------------------------------------------------------
We lift the algebra $\BDDA$ into a multi-terminal BDD algebra, whose
terminal algebra is a $\II$.  This lifting gives us the implementation
of the Cartesian product $\IICCS$ of $\II$ with $\CCS$, and allows us
to work seamlessly, and parametrically with input predicate conditions
in combination with counter conditions. This allows us to leverage
symbolic automata algorithms over the alphabet $\IICCS$. In
particular, the basic implementation of the determinization algorithm
of {\CA}s uses several underlying support algorithms of symbolic
automata modulo $\IICCS$.  In particular, during the main step of the
algorithm, mintermization is localized to powerstates, e.g., some
powerstate may have a local minterm such as
$\prodp{(\alpha_1\vee\alpha_7)}{\top_{\CCS}}$ if the different cases
do not matter locally. However, the
determinization algorithm for symbolic automata can not be used
``as is'' because it will not be able to ditinguish between
counter minterms and will overapproximate target powerstates.

